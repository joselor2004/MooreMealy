\documentclass{scrartcl}
\usepackage[a4paper, total={7in, 10in}]{geometry}
\usepackage{stmaryrd}
\usepackage{amsthm}
\usepackage{amssymb}
\usepackage{amsmath}
\usepackage{algorithm2e}
\usepackage{hyperref}
\usepackage[french]{babel}
\usepackage{color}
\usepackage{tikz}
\usetikzlibrary{automata, arrows.meta, positioning}
\usepackage{Macros}

\title{Rapport : Transducteurs et machines de Moore et Mealy}

\begin{document}

\maketitle

\RestyleAlgo{ruled}

\begin{flushleft}

\tableofcontents

\section{Généralités sur les transducteurs finis}

Un transducteur fini est une généralisation des automates finis, implémentant non pas des langages mais plutôt des
relations entre des ensembles de mots. Cela se fait à l'aide d'étiquettes supplémentaires sur chaque transition
entre deux états.

\subsection{Définition des transducteurs}

\begin{define}
    Un transducteur fini est un 6-uplet $T = (\Sigma, \Gamma, Q, I, F, \delta)$ où
    \begin{itemize}
        \item $\Sigma$ est l'alphabet d'entrée (fini)\\
        \item $\Gamma$ l'alphabet de sortie (fini)\\
        \item $Q$ l'ensemble des états (fini)\\
        \item $I \subset Q$ l'ensemble des états initiaux
        \item $F \subset Q$ l'ensemble des états finaux
        \item $\delta : Q \times \Sigma_{\varepsilon} \times \Gamma_{\varepsilon} \longrightarrow \mathcal{P}(Q)$ est
        la fonction de transition non déterministe
    \end{itemize}
\end{define}

En d'autres termes, $T$ est un automate fini non déterministe sur l'alphabet $\Sigma_{\varepsilon} \times \Gamma \cup \Sigma \times \Gamma_{\varepsilon}$
\\~\\
On peut comme pour les automates définir
$\delta^* : Q \times \Sigma^* \times \Gamma^* \longrightarrow \mathcal{P}(Q)$ par induction :
\begin{equation*}
    \begin{split}
        \delta^*_{\mid Q \times \Sigma_{\varepsilon} \times \Gamma_{\varepsilon}} &= \delta\\
        \forall (a, b) \in \Sigma_{\varepsilon} \times \Gamma_{\varepsilon},
        \forall (w, v) \in \Sigma^* \times \Gamma^*, &\forall q \in Q, \delta^*(q, wa, vb) = 
        \bigcup_{p \in \delta^*(q, w, v)} \delta(p, a, b)
    \end{split}
\end{equation*}

On définit ici la relation reconnue par un transducteur fini, de manière similaire que les
langages reconnus par les automates.
\begin{define}
    On défini $[T]$ la relation reconnue par un transducteur fini $T$, définie comme suit :
    \[ \forall (u, v) \in \Sigma^* \times \Gamma^*, u[T]v \Leftrightarrow \exists q \in I, \exists r \in F,
    r \in \delta^*(q, u, v) \]
\end{define}

\begin{figure}[h]
    \caption{Un exemple de transducteur fini reconaissant la relation de congruence modulo 2}
    \begin{center}
    Ici $\Sigma = \Gamma = \{0, 1\}$ et les transitions sont étiquettées par lettre d'entrée, lettre de sortie\\
    \vspace*{0.5cm}
    \begin{tikzpicture}[auto]
        \node (q0) [state, initial, initial text= ] {$q_0$};
        \node (q2) [state, below right = of q0] {$q_2$};
        \node (q1) [state, accepting, above right = of q2] {$q_1$};

        \path [-stealth, thick]
        (q0) edge [loop above] node {$\Sigma_{\varepsilon} \mid \Sigma_{\varepsilon}$} (q0)
        (q0) edge node {$\substack{0 \mid 0 \\ 1 \mid 1}$} (q1)
        (q0) edge [bend right] node {$\substack{0 \mid 1 \\ 1 \mid 0}$} (q2)
        (q1) edge [bend left] node {$\Sigma_{\varepsilon} \mid \Sigma_{\varepsilon}$} (q2)
        (q2) edge [loop below] node {$\Sigma_{\varepsilon} \mid \Sigma_{\varepsilon}$} (q2);
    \end{tikzpicture}
    \end{center}
\end{figure}

\vspace*{2cm}

\subsection{Relations rationnelles}

Soient $\Sigma$ et $\Gamma$ des alphabets finis. Nous allons ici étudier une classe particulière
de relations entre $\Sigma^*$ et $\Gamma^*$, que l'on peut voir comme des parties de $\Sigma^* \times \Gamma^*$,
les relations rationelles. Généralisons pour cela les opérations usuelles sur les langages 

\begin{define}
    Soient $R, R' \subset \Sigma^* \times \Gamma^*$ des relations. On pose
    \begin{itemize}
        \item $R \cdot R' = \{ (xx', yy') \mid (x, y) \in R, (x', y') \in R' \}$, la concaténation ou produit
        \item $\displaystyle R^* = \bigcup_{n \geq 0} R^n$ où $R^n$ est la concaténation répétée, et $R^0 =
        \{ (\varepsilon, \varepsilon) \}$
    \end{itemize}
\end{define}

\begin{define}
    L'ensemble des relations rationelles est défini comme le plus petit ensemble de relations stable par étoile,
    concaténation et union, et contenant $\varnothing$ et les singletons.
\end{define}

Pour être plus général, une relation rationelle est une partie rationnelle du monoïde produit
$\Sigma^* \times \Gamma^*$.\\
Comme pour les automates on a bien équivalence entre les relations reconnues par des transducteurs finis
et les relations rationelles.

\subsection{Propriétés des relations rationnelles et transducteurs finis}

De manière analogue au cas des automates finis et des langages rationnels, on dispose du théorème suivant reliant
les transducteurs finis aux relations rationelles

\begin{theorem}
    Une relation $R$ est rationelle si et seulement si elle est reconnue par un transducteur fini $T$
\end{theorem}

\begin{proof}
    La preuve est exactement du même acabi que celle pour les automates finis et langages rationnels.
\end{proof}

Cette équivalence permet de voir plus simplement certaines propriété de stabilité des transducteurs finis ou des relations
rationnelles.
\begin{prop} \label{stabRat}
    Les relations rationnelles sont stables par inverse. De plus les projections d'une relation
    rationelle sont des langages rationnels.
\end{prop}

\begin{proof}
    Si $R$ est une relation rationelle sur $\Sigma^* \times \Gamma^*$, on note $R^{-1}$ la relation inverse et
    \[ R_{\Sigma} = \{ w \in \Sigma^* \mid \exists u \in \Gamma^*, (w, u) \in R \} \]
    et de même manière $R_{\Gamma}$, les projections à droite et à gauche de $R$. Soit $T$ un transducteur fini reconaissant $R$.\\
    Si l'on considère le transducteur où les étiquettes d'entrée et de sorties sont inversées, ce dernier reconnait bien la
    relation $R^{-1}$.\\
    % Aimeric c déja stable par union par déf mdr
\end{proof}

\begin{prop} \label{imageRat}
    Soit $L \subset \Sigma^*$ rationnel. L'image par $R$ une relation rationnelle de $L$ est également régulière
\end{prop}

\begin{proof}
    Soit $\mathcal{A} = (\Sigma, Q_L, I_L, F_L, \delta_L)$ un automate reconaissant $L$. On en fait un transducteur
    $T_L = (\Sigma, \Gamma, Q_L, I_L, F_L, \delta'_L)$ où
    \[ \forall q \in Q_L, \forall a \in \Sigma, \forall b \in \Gamma_{\varepsilon}, \delta'_L(q, a, b) = \delta_L(q, a) \]
    (cela revient à coller $\Gamma_{\varepsilon}$ en étiquette de sortie de chaque transition de $\mathcal{A}$). Si on considère
    maintenant $T_R$ un transducteur fini reconaissant $R$, et que l'on considère le transducteur produit de $T_R$ et $T_L$,
    ayant pour états finaux les produits des états finaux, ce dernier reconait la relation
    \[ R' = \{ (u, v) \in R \mid u \in L \} \]
    En considérant la projection à droite, on obtient alors le langage $R(L) = \{ v \in \Gamma^* \mid \exists u \in L, (u, v) \in R \}$
    qui est bien l'image de $L$ par $R$. Donc par la proposition \ref{stabRat}, $R(L)$ est rationel
\end{proof}

\begin{prop}[Composition]
    La composition de deux relations rationnelles est rationnelle.
\end{prop}

\begin{proof}
    Si $T_1 = (\Sigma, \Gamma, Q_1, I_1, F_1, \delta_1)$ et $T_2 = (\Gamma, \Lambda, Q_2, I_2, F_2, \delta_2)$ sont des transducteurs implémentant ces deux relations,
    on construit un transducteur
    $T = (\Sigma_{\varepsilon} \times \Gamma \cup \Sigma \times \Gamma_{\varepsilon}, \Lambda, Q_1 \times Q_2, I_1 \times I_2, F_1 \times F_2, \delta_3)$
    où $\delta_3$ est définie pour $(q_1, q_2) \in Q_1 \times Q_2$ :
    \begin{equation*}
        \begin{split}
            &\forall c \in \Lambda_{\varepsilon}, \forall q_2' \in \delta_2(q_2, \varepsilon, c), (q_1, q_2') \in \delta_3((q_1, q_2),
            (\varepsilon, \varepsilon), c)\\
            &\forall a \in \Sigma, \forall q_1' \in \delta_1(q_1, a, \varepsilon), (q_1', q_2) \in \delta_3((q_1, q_2), (a, \varepsilon), \varepsilon)\\
            &\forall a \in \Sigma_{\varepsilon}, \forall b \in \Gamma, \forall c \in \Lambda_{\varepsilon},
            \forall q_1' \in \delta_1(q_1, a, b), \forall q_2' \in \delta_2(q_2, b, c), (q'_1, q'_2) \in \delta_3(q_1, q_2, (a, b), c)
        \end{split}
    \end{equation*}
    L'idée derrière $T$ est d'effectuer le calcul de $T_1$, et à chaque fois qu'une lettre est produite en sortie, on donne cette dernière en entrée
    à $T_2$ et l'on fait ainsi avancer le calcul de $T_2$
    \\~\\
    Notons que, si l'on regarde uniquement le premier élément du couple des états, on retrouve exactement le comportement de $T_1$.
    Si on regarde uniquement les deuxièmes éléments des couples des états, on trouve le deuxième transducteur, qui lit chaque lettre du deuxième élément
    du couple de sortie au moment de la transition étiquetée en sortie par cette lettre.
    \\~\\
    Ce transducteur implémente donc la relation binaire $[T]$ qui contient $((u, v), w)$ si et seulement si $(u, v)$ est dans la première relation,
    et $(v, w)$ dans la deuxième. De la même manière que dans la proposition \ref{imageRat}, on peut en intersectant $[T]$
    avec $(\Sigma^* \times \{\varepsilon\}) \times \Lambda^*$ obtenir la relation de $\Sigma^* \times \Lambda^*$
    en identifiant $\Sigma^* \times \{\varepsilon\}$ à $\Sigma^*$ (rationelle par la construction de la proposition précedente) $[T_1] \circ [T_2]$.
\end{proof}

Un exemple d'application de ces propriétés de stabilité est une preuve rapide de la rationnalité de $\operatorname*{Pref}(L)$, l'ensemble des préfixes
d'un langage rationnel $L$ : la relation $\operatorname*{Pref}$ sur $\Sigma^* \times \Sigma^*$ qui met en relation $w$ avec l'ensemble de
ses préfixes est rationnelle, reconnue par le transducteur figure \ref{transPref}. Par la propriété \ref{imageRat}, on obtient immédiatement la rationnalité de 
$\operatorname*{Pref}(L)$
\\~\\

\begin{figure}[h]
    \caption{Le transducteur reconaissant la relation $\operatorname*{Pref}$ avec $\Sigma = \{a_1, ..., a_n\}$} \label{transPref}
    \begin{center}
    \begin{tikzpicture}[auto]
        \node (q0) [state, initial, initial text= ] {$q_0$};
        \node (q1) [state, accepting, right=of q0] {$q_1$};

        \path[-stealth, thick]
        (q0) edge [loop above] node {$\substack{a_1 \mid a_1 \\ ... \\ a_n \mid a_n}$} (q0)
        (q0) edge node {$\varepsilon \mid \varepsilon$} (q1)
        (q1) edge [loop above] node {$\Sigma \mid \varepsilon$} (q1);
    \end{tikzpicture}
    \end{center}
\end{figure}

Notons toutefois que certaines propriétés de stabilité des automates finis ne se transmettent pas aux relations rationelles : c'est le cas par exemple
de l'intersection et du complémentaire. On se place sur les alphabets $\Sigma = \{a\}$ et $\Gamma = \{a, b\}$. On se donne les relations rationelles suivantes
\begin{equation*}
    \begin{split}
        R_{l} &= \{ (a^n, w) \mid w \in \Gamma^*, n \in \mathbb{N}, |w| = 2n \}\\
        R_{b} &= \{ (a^n, w) \mid w \in \Gamma^*, n \in \mathbb{N}, |w|_b = n \} 
    \end{split}
\end{equation*}
$R_l$ étant reconnue par le transducteur figure \ref{autoRl}, et $R_b$ par \ref{autoRb}.\\
On a alors que

\begin{align*}
    R &= \{ (a^n, a^n b^n) \mid n \in \mathbb{N} \}\\
    &= R_l \cap R_b \cap (\operatorname*{Pref})^{-1}
\end{align*}

alors que $R$ n'est clairement pas rationelle par la proposition \ref{stabRat} car la projection droite de $R$ n'est pas rationelle.
\\~\\
On en déduit également que des passages au complémentaires de relations ne sont généralement pas rationnelles. En effet, si
c'était le cas,
$(R_1^C \cup R_2^C)^C  = R_1 \cap R_2$ serait rationnelle puisque l'union l'est (proposition \ref{stabRat}).
\begin{figure}[h] 
    \caption{Le transducteur reconaissant la relation $R_l$}\label{autoRl}
    \begin{center}
    \begin{tikzpicture}[auto]
        \node (q0) [state, initial, initial text= ] {$q_0$};
        \node (q1) [state, right=of q0] {$q_1$};
        \node (q2) [state, accepting, below=of q0] {$q_2$};

        \path[-stealth, thick]
        (q0) edge [bend left] node {$a \mid \Gamma$} (q1)
        (q1) edge [bend left] node {$\varepsilon \mid \Gamma$} (q0)
        (q0) edge node [swap] {$\varepsilon \mid \varepsilon$} (q2);
    \end{tikzpicture}
    \end{center}
\end{figure}

\begin{figure}[h] 
    \caption{Le transducteur reconaissant la relation $R_b$}\label{autoRb}
    \begin{center}
    \begin{tikzpicture}[auto]
        \node (q0) [state, initial, accepting, initial text= ] {$q_0$};

        \path[-stealth, thick]
        (q0) edge [loop above] node {$a \mid b$} (q0)
        (q0) edge [loop below] node {$\varepsilon \mid a$} (q0);
    \end{tikzpicture}
    \end{center}
\end{figure}

\subsection{Quelques problèmes associés aux transducteurs finis}

Une question naturelle que l'on peut se poser est celle de déterminer si deux transducteur $T_1$ et $T_2$ implémentent la même relation rationelle.
De manière assez surprenante, étant donnés les résultats connus pour les automates finis, ce problème est, dans le cas général, indécidable \cite{indecEqu} (une
preuve utilise le problème de correspondance de Post).
\\~\\
On peut également se poser la question du caractère fonctionnel d'un transducteur : étant donné un transducteur fini $T$, est il possible de déterminer
si ce dernier est fonctionnel ou non, c'est à dire que la relation qu'il implémente est en réalité une fonction ? Ce problème est décidable
\cite{sakarovitch} et l'est même en temps polynômial en la taille de $T$ : la preuve utilise des notions algébriques sur les monoïdes pour
aboutir à une caractérisation de la fonctionnalité de $T$

\section{Machines de Moore}

Une machine de Moore est un cas particulier d'un transducteur fini : d'abord une telle machine est déterministe
et implémente donc non pas une relation rationelle mais une fonction, qui sera appelée rationelle. La contrainte
supplémentaire sur ces machines est que la lettre de sortie lue ne dépend que de l'état actuel de la machine.

\subsection{Définition et exemple}

On peut évidemment définir une machine de Moore en partant de la définition de transducteur et en posant des
contraintes sur la fonction de transition. Il est toutefois plus commode de la définir ainsi

\begin{define}
    Une machine de Moore $\mathcal{M}$ est un $7$-uplet $(\Sigma, \Gamma, Q, q_0, F, \delta, \lambda)$ où
    \begin{itemize}
        \item $\Sigma$ est l'alphabet d'entrée
        \item $\Gamma$ l'alphabet de sortie
        \item $Q$ l'ensemble fini des états
        \item $q_0$ l'état initial
        \item $F \subset Q$ les états finaux
        \item $\delta : Q \times \Sigma \rightarrow Q$ la fonction de transition déterministe
        \item $\lambda : Q \rightarrow \Gamma$ la fonction de sortie
    \end{itemize}
\end{define}

Un calcul de $\mathcal{M}$ sur un mot $w$ est défini de la même manière que pour un automate déterministe. On ajoute
toutefois la notion de mot produit par ce calcul, qui est le mot obtenu en concaténant les lettres obtenues
par application de $\lambda$ sur chaque état pris lors du calcul de $\mathcal{M}$ sur $w$. Plus formellement

\begin{define}
    On définit pour $q \in Q$ et $w \in \Sigma^*$ $q \star w$ récursivement :
    \[ \forall a \in \Sigma, q \star a = \lambda(\delta(q, a)) \text{ et } \forall u \in \Sigma^*,
    q \star ua = (q \star u)(\delta^*(q, u) \star a) \]
    Le mot produit par le calcul de $\mathcal{M}$ sur $w$ est défini comme $q_0 \star w$. Notons $L \subset \Sigma^*$
    le langage reconnu par l'automate déterministe contenu dans la définition de $\mathcal{M}$. La fonction rationelle
    réalisée par $\mathcal{M}$ est alors définie par
    \[ f : \begin{array}{cl}
        L &\longrightarrow \Gamma^*\\
        w &\longmapsto q_0 \star w
    \end{array} \]
\end{define}

On peut généraliser la fonction réalisée par $\mathcal{M}$ en la définissant non pas sur $L$ mais sur tout
$\Sigma^*$. Cela revient à rendre tout les états acceptants

\begin{figure}[h]
    \caption{Un exemple de machine de Moore}
    \begin{center}
        Ouais jsp j'ai envie de mettre une autre machine de Moore un peu golri
        \begin{tikzpicture}[auto, node distance = 1.5cm]

        \end{tikzpicture}
    \end{center}
\end{figure}

\vspace*{2cm}

\section{Machines de Mealy}

\subsection{Définition}

Une machine de Mealy est également un cas particulier de transducteur fini, moins restrictive à première vue que les machines de Moore.
Il s'agit d'un transducteur déterministe sans $\varepsilon$-transition.
\\~\\
Du point de vue du formalisme, elles sont définies de la même manière qu'un automate de Moore, à l'exception de la fonction de sortie
$\lambda$ qui est cette fois $\lambda : Q \times \Sigma \rightarrow \Gamma$. La fonction rationelle réalisée par une machine de Mealy
est définie de manière très similaire à celle pour les machines de Moore (il suffit juste de tenir comptre de la lettre de $\Sigma$ lue à chaque
étape du calcul)

%Tu peux faire une jolie machine de Mealy ici stv en exemple

\subsection{Equivalence avec Moore}

Le théorème suivant montre que malgré une plus grande expressivité à première vue, les machines de Mealy sont en fait aussi expressives que les machines
de Moore.

\begin{theorem}
    Pour toute machine de Mealy il existe une machine de Moore équivalente (implémentant la même fonction) et réciproquement
\end{theorem}

\begin{proof}
    Considérons d'abord le sens réciproque. Soit $\mathcal{M}_1 = (\Sigma, \Gamma, Q, q_0, F, \delta, \lambda)$ une machine de Moore et notons $f$ la
    fonction rationelle implémentée par $\mathcal{M}_1$. La machine de Mealy $\mathcal{M}_2 = (\Sigma, \Gamma, Q, q_0, F, \delta, \lambda')$ où
    $\lambda'$ est définie de la manière suivante
    \[ \forall q \in Q, \forall a \in \Sigma, \lambda'(q, a) = \lambda(\delta(q, a)) \]
    implémente la fonction $f$. Notons $f'$ la fonction implémentée par $\mathcal{M}_2$. On note d'abord que le langage rationnel $L$ reconnu par l'automate
    sous jacent est le même pour les 2 machines. En effet les deux machines effectuent les mêmes calculs (les mêmes suites d'états sont produites
    en lisant les même mots). Puis pour $w \in L$, considérons le calcul $q_0, ..., q_n$ des machines sur $w$, avec $w = w_1, ..., w_n$. On a que
    \[ f'(w) = \lambda'(q_0, w_1) ... \lambda'(q_{n-1}, w_n) = \lambda(\delta(q_0, w_1)) .... \lambda(\delta(q_{n-1}, w_n)) = q_0 \star w = f(w) \]
    \begin{figure}[h]
        \caption{Exemple de construction d'une machine de Mealy équivalente à une machine de Moore}
        \begin{center}
        Une machine de Moore à gauche et à droite la machine de Mealy correspondante
        \begin{tikzpicture}[auto, node distance=2cm]
            \node (q0) [state with output, initial, initial text= ] {$q_0$ \nodepart{lower} $a$};
            \node (q1) [state with output, accepting, below right=of q0] {$q_1$ \nodepart{lower} $b$};
            \node (q2) [state with output, above right=of q1] {$q_2$ \nodepart{lower} $a$};

            \node (q00) [state, initial, initial text=, right=of q2] {$q_0$};
            \node (q10) [state, accepting, below right=of q00] {$q_1$};
            \node (q20) [state, above right=of q10] {$q_2$};

            \path[-stealth, thick]
            (q0) edge [loop above] node {$0$} (q0)
            (q0) edge [bend left] node {$1$} (q1)
            (q1) edge node {$0$} (q2)
            (q2) edge [bend right] node {$1$} (q0)
            (q2) edge [loop above] node {$0$} (q2)
            (q1) edge [bend left] node {$1$} (q0)
            (q00) edge [loop above] node {$0 \mid a$} (q00)
            (q00) edge [bend left] node {$1 \mid b$} (q10)
            (q10) edge node {$0 \mid a$} (q20)
            (q20) edge [bend right] node {$1 \mid a$} (q00)
            (q20) edge [loop above] node {$0 \mid a$} (q20)
            (q10) edge [bend left] node {$1 \mid a$} (q00);
        \end{tikzpicture}
        \end{center}
    \end{figure}

    \vspace*{1cm}
    On montre maintenant le sens direct. Soit $\mathcal{M}_1 = (\Sigma, \Gamma, Q, q_0, F, \delta, \lambda)$ une machine de Mealy.
    On suppose quitte à rajouter un état que l'état initial est inacessible depuis n'importe quel autre état. On pose la machine de Moore
    $\mathcal{M} = (\Sigma, \Gamma, (Q \backslash q_0) \times \Gamma \cup \{q_0\}, F \times \Gamma, \delta', \lambda')$ où
    $\lambda'$ et $\delta'$ sont définis comme suit
    \begin{equation*}
        \begin{split}
            &\forall a \in \Sigma, \delta'(q_0, a) = \delta(q_0, a)\\
            &\forall q \in Q \backslash q_0, \forall b \in \Gamma, \forall a \in \Sigma, \delta'(q, b, a) = (\delta(q, a), \lambda(q, a))\\
            &\forall q \in Q \backslash q_0, \forall b \in \Gamma, \lambda'(q, b) = b
        \end{split}
    \end{equation*}
    Notons d'abord que $w \in \Sigma^*$ est reconnu par l'automate d'entrée de $\mathcal{M}_1$, si et seulement si il l'est par l'automate d'entrée de 
    $\mathcal{M}$ : en effet, un calcul $q_0, ..., q_n$ dans le premier automate correspond à un calcul $q_0, (q_1, b_1), ..., (q_n, b_n)$
    dans le second automate avec $b_1, ..., b_n \in \Gamma$ et réciproquement. Soit $L$ le langage reconnu par les deux automates d'entrées,
    $f$ la fonction implémentée par $\mathcal{M}_1$ et $F$ celle implémentée par $\mathcal{M}$. Pour $w \in L$, $w = w_1 ... w_n$
    et produisant le calcul $q_0, (q_1, b_1), ..., (q_n, b_n)$ dans $\mathcal{M}$
    \begin{align*}
        F(w) &= q_0 \star w\\
        &= \lambda'(q_1, b_1) ... \lambda'(q_n, b_n)\\
        &= b_1 ... b_n\\
        &= \lambda(q_0, w_1) ... \lambda(q_{n-1}, w_n)\\
        &= f(w)
    \end{align*}
    Donc les deux transducteurs sont bien équivalents

    %Mettre dessin
\end{proof}

\subsection{Propriétés algébriques}

\bibliographystyle{plain}
\bibliography{Rapport}

\end{flushleft}
\end{document}